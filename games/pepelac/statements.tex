\documentclass[a4paper,12pt]{article}
\usepackage[utf8]{inputenc}
\usepackage[russian]{babel}
\pagestyle{empty}
\begin{document}

\begin{center}
{\small\textsc{Параллель Промышленного Программирования «П» Представляет Первенство Программистов По Патрономании}}
\vskip 1pt \hrule \vskip 3pt

\end{center}
\begin{abstract}
Участникам соревнования предстоит реализовать алгоритм, который будет руководить ботом, перемещающимся по полю в поисках патронов и сражающимся с другими игроками. Цель каждого игрока ~--- как можно дольше оставаться в живых.
\end{abstract}
\subsection{Правила игры}
Имеется поле $N\times N$, на котором случайным образом расположены $B$ патронов и $P$ игроков. В начале игры каждый игрок получает информацию о расположении патронов и других участников раунда. Игроки ходят по очереди. В процессе хода игрок может передвинуться на одну клетку вверх, вниз, влево, вправо или остаться на месте. Если в клетке, куда он передвинулся, находится патрон, то он сразу же поднимается игроком. Если в соседней по стороне с игроком клетке находится другой игрок, происходит сражение: игрок с меньшим количеством патронов умирает, а у игрока с б\'{о}льшим количеством патронов вычитаются патроны погибшего участника. В случае одинакового количества патронов, они обнуляются и начинается следующий ход (оба игрока выживают). Если в соседних по стороне клетках находится более двух игроков, битву начинает тот, кто сходил последним и сражается со всеми соперниками по часовой стрелке, начиная сверху. С некоторого хода у игрового поля по спирали исчезает одна клетка: первой исчезает клетка в левом верхнем углу, а последней ~--- клетка в самом центре.
\begin{flushleft}
Игрок выбывает из игры, если его бот:
\begin{itemize}
\item погиб в результате сражения с ботом другого игрока
\item попытался выйти за пределы поля
\item не уложился в лимит времени или памяти:
TL = 1 секунда на 50 ходов, ML = 256 мегабайт
\item попытался совершить действия, которые тестирующая система сочла небезопасными
\item находился на клетке, которая исчезла
\item завершил работу до окончания игры
\end{itemize}
\end{flushleft}
\subsection{Определение победителя}
Целью игры является продержаться на игровом поле как можно дольше. Соответственно, победителем будет признан игрок, который на момент исчезновения игрового поля совершил больше всего ходов.

\subsection{Формат входного файла}
Для удобства работы будем считать, что у левой верхней клетки координаты $(1;1)$, а у правой нижней ~--- $(N;N)$. В начале игры в первой строке задается размер поля $N$ ($40 \leq N \leq 50$).
В процесс игры в начале каждого хода задается количество игроков $P$ ($1 \leq P \leq 2$), количество патронов $B$ ($1 \leq B \leq N\times N - P$) и время, оставшееся до Армагеддона $K$ ($-N\times N \leq K \leq 10$, пока число положительное, клетки не исчезают ~--- как только число станет отрицательным, Армагеддон начнется).
Во второй строке задаются координаты участника ($x_1$, $y_1$) и количество патронов $b_1$.
В следующих $P-1$ строках построчно задаются координаты других игроков вида ($x_i$, $y_i$) и количество патронов у участника $B_i$.
В последних $B$ строках задаются координаты патронов вида ($x_k$, $y_k$).

\subsection{Формат выходного файла}
Программа участника должна вернуть направление, в котором совершает ход: <<UP>>, если нужно передвинуться на одну клетку вверх, <<DOWN>> ~--- на одну клетку вниз, <<LEFT>> ~--- на одну клетку влево, <<RIGHT>> ~--- на одну клетку вправо или <<STAND>>, если нужно остаться на месте.

\subsection{Взаимодействие с турнирной системой}
Затем программа-решение начинает взаимодействие с турнирной системой в соответствии со следующим протоколом:
Программа выводит в стандартный поток вывода одну строку, описывающую ход бота (смотрите формат вывода в разделе \textbf{Формат выходного файла}). Вывод должен завершаться переводом строки и сбросом буфера потока
вывода. Для этого используйте
\begin{itemize}
\item flush(output) в паскале или Delphi;
\item fflush(stdout) или cout.flush() в С/C++;
\item Console.out.flush() в Visual Basic.
\item sys.stdout.flush() в Python.
\end{itemize}
После этого программа должна считать из стандартного потока ввода ответ тестирующей системы, описанный в разделе \textbf{Формат входного файла} (повторно выводится вся информация, кроме размера поля ~--- он указывается только в начале игры).
\end{document}



