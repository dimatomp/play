\documentclass[a4paper,12pt]{article}
\usepackage[utf8]{inputenc}
\usepackage[russian]{babel}
\pagestyle{empty}
\begin{document}

\begin{center}
{\small\textsc{Параллель Промышленного Программирования «П» Представляет Первенство Программистов По Патрономании}}
\vskip 1pt \hrule \vskip 3pt

\end{center}
\begin{abstract}
Участникам соревнования предстоит реализовать алгоритм, который будет руководить ботом, перемещающимся по полю в поисках патронов и сражающимся с другими игроками. Цель каждого игрока — как можно дольше оставаться в живых.
\end{abstract}
\subsection{Правила игры}
Имеется поле $N$$*$$N$, на котором случайным образом расположены $P$ патронов и $X$ игроков. В начале игры каждый игрок получает информацию о расположении патронов и других участников раунда. Игроки ходят по очереди. В процессе хода игрок может передвинуться на одну клетку вверх, вниз, влево, вправо или остаться на месте. Если в клетке, куда он передвинулся, находится патрон, то он сразу же поднимается игроком. Если в соседней по стороне с игроком клетке находится другой игрок, происходит сражение: игрок с меньшим количеством патронов умирает, а у игрока с большим количеством патронов вычитаются патроны погибшего участника. В случае одинакового количества патронов, они обнуляются и начинается следующий ход (оба игрока выживают). Если в соседних по стороне клетках находится более двух игроков, битву начинает тот, кто сходил последним и сражается со всеми соперниками по часовой стрелке, начиная сверху. С некоторого (неизвестного участникам) хода у игрового поля по спирали исчезает одна клетка: первой исчезает клетка в правом верхнем углу, а последней — клетка в самом центре.
\begin{flushleft}
Игрок выбывает из игры, если его бот:
\begin{itemize}
\item погиб в результате сражения с ботом другого игрока
\item попытался выйти за пределы поля
\item не уложился в лимит времени или памяти
\item попытался совершить действия, которые тестирующая система сочла небезопасными
\item находился на клетке, которая исчезла
\item завершил работу до окончания игры
\end{itemize}
\end{flushleft}
\subsection{Определение победителя}
Целью игры является продержаться на игровом поле как можно дольше. Соответственно, победителем будет признан игрок, который на момент исчезновения игрового поля совершил больше всего ходов.

\subsection{Формат входного файла}
В первой строке задается размер поля $N$, количество патронов $P$, количество игроков $S$ и количество исчезнувших клеток $K$.
Во второй строке задаются координаты участника ($x_1$, $y_1$) и количество патронов $p_1$.
В следующих $S-1$ строках построчно задаются координаты других игроков вида ($x_i$, $y_i$) и количество патронов у участника $p_i$.
В последних $P$ строках задаются координаты патронов вида ($x_k$, $y_k$).

\subsection{Формат выходного файла}
Программа участника должна вернуть координаты клетки ($x$, $y$), в которые совершает ход. Если будут возвращены текущие координаты, бот останется на месте.
\end{document}